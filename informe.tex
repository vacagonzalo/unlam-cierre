\documentclass[
    11pt,
    spanish,
    a4paper
]{article}
\usepackage[utf8]{inputenc}
\usepackage[spanish]{babel}
\usepackage{authoraftertitle}
\usepackage{booktabs}
\usepackage{caption}
\usepackage{float}
\usepackage{graphicx}
\usepackage{listings}
\usepackage{verbatim}

\def\doctype{INFORME DE INVESTIGACIÓN}
\title{Softwares de simulación para Tecnología electrónica y otras cátedras de ingeniería}
\author{Gonzalo Nahuel Vaca}

\begin{document}

\makeatletter
\begin{titlepage}
	\begin{center}
		\vspace*{1cm}

		\Huge
		\textbf{\doctype}
		\vspace{0.5cm}

		\LARGE
		\@title
		\vspace{0.5cm}

		\textbf{Compatibilidad electromagnética}

		\vspace{1.5cm}

		\textbf{\@author}

		\vspace{1.5cm}

		\includegraphics[width=0.4\textwidth]{img/logoUNLaM.png}

		\vfill
		Universidad Nacional de la Matanza\\
		Argentina\\
		\today
	\end{center}
\end{titlepage}
\makeatother
\newpage

\section*{Resumen}

Este documento trata el estado de avance de la investigación del \emph{software} \emph{openEMS}.

\section*{Introducción}

\emph{openEMS} es un programa que soluciona simulaciones electromagnéticas.
Su núcleo utiliza el método de \emph{Finite-diffrecence time-domain} (FDTD) y se implementó en el lenguaje de programación \emph{C}.

Además de su núcleo tiene los siguientes módulos:

\begin{itemize}
\item Biblioteca \emph{QSCXCAD}: componente que permite representar geometrías.
\item \emph{AppCSXCAD}: interfaz gráfica para la biblioteca \emph{QSCXCAD}.
\item Biblioteca\emph{CSXCAD}: permite describir objetos geométricos y sus propiedades físicas y no-físicas.
\item \emph{CTB}: herramienta para circuitos en \emph{Maltab/Octave}.
\item \emph{fparser}: analizador de funciones para el lenguaje de programación \emph{C++}.
\item \emph{hyp2mat}: ofrece la posibilidad de importar archivos \emph{Hyperlynx Boardsim} a \emph{openEMS} en \emph{Matlab/Octave}.
\end{itemize}

\section*{Interfaz de programación de aplicaciones}

Existen dos interfaces de programación de aplicaciones (API): \emph{Matlab/Octave} y \emph{Python 3}.

\emph{Matlab/Octave} es la API original del programa, sin embargo tiene poco mantenimiento y los módulos relevantes para la simulación de compatibilidad electromagnética se encuentran en un estado no funcional.

La nueva API es \emph{Python 3}, sus módulos se encuentran en mantenimiento activo y continuo desarrollo.
La única desventaja es que tienen un grado de maduración menor y ofrecen menos funcionalidades que la API original.

\section*{Evaluación de la interfaz \emph{Python 3}}

\section*{Diseño de simulaciones}

\end{document}
